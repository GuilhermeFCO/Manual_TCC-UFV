% Options for packages loaded elsewhere
\PassOptionsToPackage{unicode}{hyperref}
\PassOptionsToPackage{hyphens}{url}
%
\documentclass[
]{article}
\usepackage{lmodern}
\usepackage{amssymb,amsmath}
\usepackage{ifxetex,ifluatex}
\ifnum 0\ifxetex 1\fi\ifluatex 1\fi=0 % if pdftex
  \usepackage[T1]{fontenc}
  \usepackage[utf8]{inputenc}
  \usepackage{textcomp} % provide euro and other symbols
\else % if luatex or xetex
  \usepackage{unicode-math}
  \defaultfontfeatures{Scale=MatchLowercase}
  \defaultfontfeatures[\rmfamily]{Ligatures=TeX,Scale=1}
\fi
% Use upquote if available, for straight quotes in verbatim environments
\IfFileExists{upquote.sty}{\usepackage{upquote}}{}
\IfFileExists{microtype.sty}{% use microtype if available
  \usepackage[]{microtype}
  \UseMicrotypeSet[protrusion]{basicmath} % disable protrusion for tt fonts
}{}
\makeatletter
\@ifundefined{KOMAClassName}{% if non-KOMA class
  \IfFileExists{parskip.sty}{%
    \usepackage{parskip}
  }{% else
    \setlength{\parindent}{0pt}
    \setlength{\parskip}{6pt plus 2pt minus 1pt}}
}{% if KOMA class
  \KOMAoptions{parskip=half}}
\makeatother
\usepackage{xcolor}
\IfFileExists{xurl.sty}{\usepackage{xurl}}{} % add URL line breaks if available
\IfFileExists{bookmark.sty}{\usepackage{bookmark}}{\usepackage{hyperref}}
\hypersetup{
  pdftitle={Manual TCC-UFV},
  pdfauthor={Guilherme Fernandes Castro de Oliveira},
  hidelinks,
  pdfcreator={LaTeX via pandoc}}
\urlstyle{same} % disable monospaced font for URLs
\usepackage{longtable,booktabs}
% Correct order of tables after \paragraph or \subparagraph
\usepackage{etoolbox}
\makeatletter
\patchcmd\longtable{\par}{\if@noskipsec\mbox{}\fi\par}{}{}
\makeatother
% Allow footnotes in longtable head/foot
\IfFileExists{footnotehyper.sty}{\usepackage{footnotehyper}}{\usepackage{footnote}}
\makesavenoteenv{longtable}
\usepackage{graphicx,grffile}
\makeatletter
\def\maxwidth{\ifdim\Gin@nat@width>\linewidth\linewidth\else\Gin@nat@width\fi}
\def\maxheight{\ifdim\Gin@nat@height>\textheight\textheight\else\Gin@nat@height\fi}
\makeatother
% Scale images if necessary, so that they will not overflow the page
% margins by default, and it is still possible to overwrite the defaults
% using explicit options in \includegraphics[width, height, ...]{}
\setkeys{Gin}{width=\maxwidth,height=\maxheight,keepaspectratio}
% Set default figure placement to htbp
\makeatletter
\def\fps@figure{htbp}
\makeatother
\setlength{\emergencystretch}{3em} % prevent overfull lines
\providecommand{\tightlist}{%
  \setlength{\itemsep}{0pt}\setlength{\parskip}{0pt}}
\setcounter{secnumdepth}{5}
\usepackage{booktabs}
\usepackage[]{natbib}
\bibliographystyle{apalike}

\title{Manual TCC-UFV}
\author{Guilherme Fernandes Castro de Oliveira}
\date{Última atualização: 11/09/2020}

\begin{document}
\maketitle

{
\setcounter{tocdepth}{2}
\tableofcontents
}
\hypertarget{introduuxe7uxe3o}{%
\section*{Introdução}\label{introduuxe7uxe3o}}
\addcontentsline{toc}{section}{Introdução}

\hypertarget{instalauxe7uxe3o-windows}{%
\section{Instalação Windows}\label{instalauxe7uxe3o-windows}}

Para utilizarmos o padrão devemos instalar alguns softwares, eles são o Miktek, o R e o Rstudio. Segue abaixo cada um deles na ordem que devem ser instalados.

\hypertarget{miktek}{%
\subsubsection{Miktek}\label{miktek}}

Para fazer o download do Miktek no Windows, é preciso fazer o download do instalador .exe, no link que segue abaixo você irá ser redirecionado a página de download do Miktek, na página que você sera levado, é só clicar no botão azul \textbf{\emph{Download}}. Após o link, seguirá uma screenshot para ajudar no download do Miktek.

\begin{itemize}
\tightlist
\item
  \href{https://miktex.org/download}{Download MiKTeX} \textbf{(Clique segurando o `ctrl')}
\end{itemize}

Após o download você deve entrar na pasta onde você fez o download, e executar o instalador do Miktek. Segue abaixo screenshots para ajudar na instalação do miktek.

\hypertarget{r}{%
\subsubsection{R}\label{r}}

~Após a instalação do Miktek, você deve fazer o download do executável do software R para poder instala-ló no seu windows. Para o download do R, segue o link abaixo para a página de download, você deve clicar em \textbf{\emph{Download R for Windows}} na página onde você será levado.

\begin{itemize}
\tightlist
\item
  \href{https://cran-r.c3sl.ufpr.br/}{Download R para Windows} \textbf{(Clique segurando o `ctrl')}
\end{itemize}

\hypertarget{rstudio}{%
\subsubsection{Rstudio}\label{rstudio}}

Após a instação do miktek, temos a instalação do RStudio. \textbf{Clique abaixo segurando o `ctrl'.}

\begin{itemize}
\tightlist
\item
  \href{https://rstudio.com/products/rstudio/download/\#download}{Download RStudio}
\end{itemize}

\hypertarget{instalauxe7uxe3o-linux}{%
\section{Instalação Linux}\label{instalauxe7uxe3o-linux}}

Para utilizarmos o padrão devemos instalar alguns softwares, eles são o Miktek, o R e o Rstudio. Segue abaixo cada um deles na ordem que devem ser instalados.

\hypertarget{miktek-1}{%
\subsubsection{Miktek}\label{miktek-1}}

\hypertarget{r-1}{%
\subsubsection{R}\label{r-1}}

\hypertarget{rstudio-1}{%
\subsubsection{Rstudio}\label{rstudio-1}}

\hypertarget{utilizando}{%
\subsection{Utilizando}\label{utilizando}}

  \bibliography{book.bib,packages.bib}

\end{document}
